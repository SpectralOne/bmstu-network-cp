\anonsection{ПРИЛОЖЕНИЕ A}\label{app:server}

\begin{lstlisting}[label=lst:s1,caption=Исходный код сервера, language=c]
	#define _GNU_SOURCE
	
	#include <arpa/inet.h>
	#include <fcntl.h>
	#include <magic.h>
	#include <netinet/in.h>
	#include <pthread.h>
	#include <stdio.h>
	#include <stdlib.h>
	#include <string.h>
	#include <sys/socket.h>
	#include <sys/types.h>
	#include <unistd.h>
	
	#include "log.h"
	
	#define INDEX_HTML_PATH "public/index.html"
	#define FAVICON_PATH "public/favicon.ico"
	
	#define MAX_THREADS 10
	#define MAX_REQUEST_SIZE 4096
	#define MAX_RESPONSE_SIZE 4096
	#define MAX_BUFFER_SIZE 4096
	
	#define RESPONSE_HEADERS \
	"HTTP/1.1 200 OK\r\nContent-Length: %lld\r\nContent-Type: %s\r\n\r\n"
	#define FORBIDDEN "HTTP/1.1 403 Forbidden\r\n\r\n"
	#define NOT_FOUND "HTTP/1.1 404 Not Found\r\n\r\n"
	#define NOT_ALLOWED "HTTP/1.1 405 Method Not Allowed\r\n\r\n"
	
	typedef struct {
		int client_sockfd;
		struct sockaddr_in client_addr;
	} request_t;
	int send_file_with_response(FILE* file, const char* path, char* response,
\end{lstlisting}

\begin{lstlisting}[label=lst:s2,caption=Исходный код сервера, language=c]
request_t* req, struct magic_set* magic) {
	log_trace("Processing file at path: %s", path);
	
	fseek(file, 0, SEEK_END);
	long long int file_size = ftell(file);
	fseek(file, 0, SEEK_SET);
	
	log_debug("File size = %lld", file_size);
	
	sprintf(response, RESPONSE_HEADERS,
	(long long int)strlen(response) + file_size, magic_file(magic, path));
	
	write(req->client_sockfd, response, strlen(response));
	
	log_trace("Sending file at path: %s", path);
	
	while (file_size > 0) {
		uint8_t buffer[MAX_BUFFER_SIZE] = {0};
		size_t bytes_readed = fread(buffer, 1, MAX_BUFFER_SIZE, file);
		file_size -= bytes_readed;
		write(req->client_sockfd, buffer, bytes_readed);
	}
	log_trace("File sent");
	return 0;
}

void request_handler(request_t* req) {
	log_trace("Thread found, started request handling activity");
	
	char request[MAX_REQUEST_SIZE] = {0};
	char response[MAX_RESPONSE_SIZE] = {0};
	char dir_buff[MAX_BUFFER_SIZE] = {0};
	
	struct magic_set* magic = magic_open(MAGIC_MIME | MAGIC_CHECK);
	magic_load(magic, NULL);
	
	read(req->client_sockfd, request, sizeof(request));
\end{lstlisting}

\begin{lstlisting}[label=lst:s3,caption=Исходный код сервера, language=c]
char method[10] = {0};
char path[256] = {0};
sscanf(request, "%s %s", method, path);

log_debug("Method = %s Path = %s", method, path);

if (strcmp(method, "GET") != 0 && strcmp(method, "HEAD") != 0) {
	sprintf(response, NOT_ALLOWED);
	
	log_info("Method %s not allowed", method);
	
	write(req->client_sockfd, &response, strlen(response));
	close(req->client_sockfd);
	free(req);
	return;
}

if (strcmp(path, "/") == 0 && strcmp(method, "GET") == 0) {
	log_info("Returning default HTML");
	
	FILE* index_html = fopen(INDEX_HTML_PATH, "rb");
	
	send_file_with_response(index_html, INDEX_HTML_PATH, response, req, magic);
	
	fclose(index_html);
	
	log_info("/ request succeded");
	return;
}

if (strcmp(path, "/favicon.ico") == 0 && strcmp(method, "GET") == 0) {
	log_info("Returning favicon");
	
	FILE* icon = fopen(FAVICON_PATH, "rb");
	
	send_file_with_response(icon, FAVICON_PATH, response, req, magic);
\end{lstlisting}

\begin{lstlisting}[label=lst:s4,caption=Исходный код сервера, language=c]
	
fclose(icon);

log_info("/favicon.ico request succeded");
return;
}

getcwd(dir_buff, MAX_BUFFER_SIZE);
char full_path[MAX_BUFFER_SIZE] = {0};
strcpy(full_path, dir_buff);
strcat(full_path, path);

log_debug("Full path = %s", full_path);

if (strstr(full_path, "..") != NULL) {
sprintf(response, FORBIDDEN);

log_info("Path forbidden: %s", full_path);

write(req->client_sockfd, &response, strlen(response));
close(req->client_sockfd);
free(req);
return;
}

FILE* file = fopen(full_path, "rb");
if (file == NULL) {
sprintf(response, NOT_FOUND);

log_info("File %s not found", full_path);

write(req->client_sockfd, &response, strlen(response));
close(req->client_sockfd);
free(req);
return;
}

if (strcmp(method, "GET") == 0) {
send_file_with_response(file, full_path, response, req, magic);
} else if (strcmp(method, "HEAD") == 0) {
\end{lstlisting}

\begin{lstlisting}[label=lst:s5,caption=Исходный код сервера, language=c]
sprintf(response, RESPONSE_HEADERS, (long long int)strlen(response),
magic_file(magic, full_path));
write(req->client_sockfd, &response, strlen(response));

log_info("HEAD request succeded");
}

close(req->client_sockfd);

fclose(file);
free(req);
}

int main(void) {
setbuf(stdout, NULL);

int server_sockfd;
int client_sockfd;
struct sockaddr_in server_addr;
struct sockaddr_in client_addr;
socklen_t client_addr_len;

server_sockfd = socket(AF_INET, SOCK_STREAM, 0);
if (server_sockfd < 0) {
log_error("Failed to create socket");
exit(EXIT_FAILURE);
}

server_addr.sin_family = AF_INET;
server_addr.sin_addr.s_addr = INADDR_ANY;
server_addr.sin_port = htons(8080);

if (bind(server_sockfd, (struct sockaddr*)&server_addr, sizeof(server_addr)) <
0) {
log_error("Failed to bind socket");
exit(EXIT_FAILURE);
}
\end{lstlisting}

\begin{lstlisting}[label=lst:s6,caption=Исходный код сервера, language=c]
if (listen(server_sockfd, SOMAXCONN) < 0) {
	log_error("Failed to listen");
	exit(EXIT_FAILURE);
}
pthread_t threads[MAX_THREADS];
for (int i = 0; i < MAX_THREADS; ++i) {
	threads[i] = 0;
}
log_info("Started at 0.0.0.0:8080");
fd_set rset;
FD_ZERO(&rset);
while (1) {
	FD_SET(server_sockfd, &rset);
	pselect(server_sockfd + 1, &rset, NULL, NULL, NULL, NULL);
	if (!FD_ISSET(server_sockfd, &rset)) continue;
	client_addr_len = sizeof(client_addr);
	client_sockfd =	accept(server_sockfd, (struct sockaddr*)&client_addr, &client_addr_len);
	if (client_sockfd < 0) {
		log_error("Failed to accept");
		continue;
	}
	request_t* req = (request_t*)malloc(sizeof(request_t));
	if (req == NULL) {
		log_fatal("Failed to allocate memmory for request");
		continue;
	}
	req->client_sockfd = client_sockfd;
	req->client_addr = client_addr;
	log_trace("Request accepted, started thread lookup");	
	for (int i = 0; i < MAX_THREADS; ++i) {
		if (threads[i] == 0 || (pthread_tryjoin_np(threads[i], NULL) == 0)) {
			pthread_create(&threads[i], NULL, (void* (*)(void*))request_handler,
			(void*)req);
			break;
		}}}
return EXIT_SUCCESS;
}
\end{lstlisting}

\clearpage
\specsection{ПРИЛОЖЕНИЕ Б} \label{app:logh}
\begin{lstlisting}[label=code:logh1, caption={Заголовочный файл логгера}]
#ifndef LOG_H
#define LOG_H

#include <stdarg.h>
#include <stdbool.h>
#include <stdio.h>
#include <time.h>
typedef struct {
	va_list ap;
	const char *fmt;
	const char *file;
	struct tm *time;
	void *udata;
	const int line;
	const int level;
} log_event_t;
enum { LOG_TRACE, LOG_DEBUG, LOG_INFO, LOG_WARN, LOG_ERROR, LOG_FATAL };
#define log_trace(...) log_log(LOG_TRACE, __FILE__, __LINE__, __VA_ARGS__)
#define log_debug(...) log_log(LOG_DEBUG, __FILE__, __LINE__, __VA_ARGS__)
#define log_info(...) log_log(LOG_INFO, __FILE__, __LINE__, __VA_ARGS__)
#define log_warn(...) log_log(LOG_WARN, __FILE__, __LINE__, __VA_ARGS__)
#define log_error(...) log_log(LOG_ERROR, __FILE__, __LINE__, __VA_ARGS__)
#define log_fatal(...) log_log(LOG_FATAL, __FILE__, __LINE__, __VA_ARGS__)

void log_set_level(const int level);
void log_set_quiet(const bool enable);

void log_log(int level, const char *file, int line, const char *fmt, ...);
#endif
\end{lstlisting}

\clearpage

\specsection{ПРИЛОЖЕНИЕ В} \label{app:logc}

\begin{lstlisting}[label=code:logc1, caption={Исходный код логгера}]
#include "log.h"
static struct {
	void *udata;
	int level;
	bool quiet;
} Logger;

static const char *level_strings[] = {"TRACE", "DEBUG", "INFO",
	"WARN",  "ERROR", "FATAL"};

static const char *level_colors[] = {"\x1b[94m", "\x1b[36m", "\x1b[32m",
	"\x1b[33m", "\x1b[31m", "\x1b[35m"};

static void stdout_callback(log_event_t *ev) {
	char buf[16];
	buf[strftime(buf, sizeof(buf), "%H:%M:%S", ev->time)] = '\0';
	fprintf(ev->udata, "%s %s%-5s\x1b[0m \x1b[90m%s:%d:\x1b[0m ", buf,
	level_colors[ev->level], level_strings[ev->level], ev->file,
	ev->line);
	vfprintf(ev->udata, ev->fmt, ev->ap);
	fprintf(ev->udata, "\n");
	fflush(ev->udata);
}

void log_set_level(const int level) { Logger.level = level; }

void log_set_quiet(const bool enable) { Logger.quiet = enable; }

static void init_event(log_event_t *ev, void *udata) {
	if (!ev->time) {
		time_t t = time(NULL);
		ev->time = localtime(&t);
	}
	ev->udata = udata;
}
\end{lstlisting}


\begin{lstlisting}[label=code:logc2, caption={Исходный код логгера}]
void log_log(int level, const char *file, int line, const char *fmt, ...) {
	log_event_t ev = {
		.fmt = fmt,
		.file = file,
		.line = line,
		.level = level,
	};
	
	if (!Logger.quiet && level >= Logger.level) {
		init_event(&ev, stderr);
		va_start(ev.ap, fmt);
		stdout_callback(&ev);
		va_end(ev.ap);
	}
}
\end{lstlisting}
